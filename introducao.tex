% ----------------------------------------------------------
% Exemplo de capítulo sem numeração, mas presente no Sumário
\chapter[Introduction]{Introduction}
\addcontentsline{toc}{chapter}{Introduction}
% ----------------------------------------------------------

Railways are one of the most important modes of transportation worldwide. It is known for its efficiency, cost-effectiveness, and ability to move large volumes of goods and passengers over long distances. According to the Internacional Railway Union (UIC), in 2024 railways transported more than 24 billion passengers and over 9 billion tons of freight \cite{Railway_data_2024}. Given this scale, it is crucial to ensure the safety and reliability of the railway infrastructure, which can be achieved through regular maintenance and monitoring of the tracks.  

The monitoring of railway tracks aims to identify potential irregularities and defects that require maintenance. Track irregularities are deviations from the ideal geometry of the track that can affect the comfort and safety of train operations. They are caused by various factors, such as wear and tear, that degrade the track condition. In severe cases, they can cause accidents, such as a derailment. Traditionally, monitoring of railway tracks has been done using a Track Geometry Car (TCG) that measures directly the track geometry. This process is usually done monthly, due to its high operational cost and time requirements, as it requires the railway operation to halt services during inspections that can take several hours depending on the length of the track measured.

An alternative approach for monitoring railway track is to use an Instrumented Railway Vehicle (IRV), equipped with a series of sensors, that collects data while the train is in operation. The IRV measures the dynamic response of the wagon, for example, the acceleration in the axlebox, due to track excitations. This method is less expensive and can be done more frequently, as it collects data from a moving vehicle. However, it comes with a trade-off as the collected data is not the geometry of the track, but the dynamic response of the wagon, which needs to be mapped to track irregularities.

Several techniques have been proposed for this maping, such as the Kalman filter or traditional machine learning methods. These methods aim to reconstruct the track irregularities from the data collected by the IRV, but often face limitations when dealing with complex and noisy data. To deal with this, deep learning techniques have been increasingly applied, as they offer more robust solutions to noisy data.

This thesis proposes a deep learning model capable of reconstructing railway track irregularities from acceleration measurements obtained through simulations and real-world data. The study considers the effects of train velocity and wagon mass on the measured responses, proposing correction methodologies to minimize their influence. By integrating validated multibody simulations with advanced learning algorithms, the proposed approach aims to develop a machine learning model that is capable of being trained in simulated data and work with real data without the need of retraining. 

This work is organized as follows: Section \ref{sec2_revisao_lit} presents a literature review on the topic of railway track monitoring, focusing on the methods used to reconstruct track irregularities from acceleration data. Section \ref{sec-methodology} describes the methodology used in this study, including the defined parameters of the simulation and the machine learning model used. Section \ref{sec-Results} presents the results obtained, and Section \ref{sec-conclusion} concludes the work, discussing its contributions and future work.
