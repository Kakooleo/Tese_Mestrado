% ----------------------------------------------------------
% Exemplo de capítulo sem numeração, mas presente no Sumário
\chapter[Introduction]{Introduction}
\addcontentsline{toc}{chapter}{Introduction}
% ----------------------------------------------------------

Railways are one of the most important modes of transportation worldwide. It is known for its efficiency, cost-effectiveness, and ability to move large volumes of goods and passengers over long distances. According to the Internacional Railway Union (UIC), in 2024 railways transported more than 24 billion passengers and over 9 billion tons of freight \cite{UIC_2024}. Given this scale, it is crucial to ensure the safety and reliability of the railway infrastructure, which can be achieved through regular maintenance and monitoring of the tracks.  

The monitoring of railway tracks aims to identify potential defects that require maintenance. Track irregularities are deviations from the ideal geometry of the track that can affect the comfort and safety of train operations. They are caused by various factors, such as wear and tear, that degrade the track condition. In severe cases, they can cause accidents, such as derailments. Traditionally, monitoring of railway tracks has been done using a Track Geometry Cars (TCGs) that measures directly the track geometry. This process is usually done monthly, due to its high operational cost and time requirements, as it requires the railway operation to halt services during inspections which can take several hours depending on the length of the track measured.

An alternative approach for monitoring railway track is to use an Instrumented Railway Vehicle (IRV), equipped with a series of sensors that collects data while the train is in operation. The IRV measures the dynamic response of the wagon due to track excitations. This method is less expensive and can be done more frequently, as it collects data from a moving vehicle. However, it comes with a trade-off as the collected data is not the geometry of the track, but the dynamic response of the wagon, which needs to be mapped to track irregularities. This mapping, however, is not straightforward, as the measured dynamic of the system is heavily influenced by various factors, such as train speed, track condition, wagon characteristics, and the inherent noise of real-world data. Therefore, IRV data must be carefully processed and analyzed to extract meaningful information about the track condition.

Several techniques have been proposed to try solving this mapping problem, such as the Kalman filter or traditional machine learning methods. These methods aim to reconstruct the track irregularities from the data collected by the IRV, but often face limitations when dealing with complex and noisy data. More recently, to deal with the complexity of the problem, deep learning methods have been proposed as a solution to reconstruct track irregularities. These models use advanced architectures to learn the correlations and patterns within the data. However, most of the proposed models don't consider the influence of external factors, such as train speed and wagon mass, on the measured responses. This can reduce the robustness of the method, as it might not be able to generalize well to different operational conditions.

This thesis proposes a deep learning model capable of reconstructing railway track irregularities from acceleration measurements. Data from multibody simulation was used to create the model and later on validated on real-world data. The study considers the effects of train velocity and wagon mass on the measured responses, proposing a correction method that normalizes the acceleration under different operational conditions. By combining validated multibody simulations with advanced learning algorithms, the proposed approach aims to develop a robust framework for railway track monitoring.

This work is organized as follows: Section \ref{sec2_revisao_lit} presents a literature review on the topic of railway track monitoring, focusing on the methods used to reconstruct track irregularities from acceleration data. Section \ref{sec-methodology} describes the methodology used in this study, including the defined parameters of the simulation and the machine learning model used. Section \ref{sec-Results} presents the results obtained, and Section \ref{sec-conclusion} concludes this thesis, discussing its contributions and future work.
