% -------------------------------------------------------------
%  RESUMOS
\setlength{\absparsep}{18pt} % ajusta o espaçamento dos parágrafos do resumo
% -------------------------------------------------------------
% ATENÇÃO: o ambiente 'otherlanguage*' deve ser usado para o resumo que não está na
% língua vernácula do trabalho, com a respectiva opção linguística do pacote 'babel'.
% -------------------------------------------------------------
% resumo em PORTUGUÊS (OBRIGATÓRIO)
\begin{resumo}[Resumo]
 \begin{otherlanguage*}{brazil}
    Esta tese propõe uma abordagem de aprendizado profundo para a reconstrução de irregularidades da via a partir de dados de aceleração. O estudo investiga primeiramente a influência da velocidade e da massa do vagão nas medições de aceleração, com o objetivo de normalizar os dados para diferentes condições operacionais. Em seguida, um modelo de aprendizado de máquina será treinado utilizando dados simulados para reconstruir as irregularidades da via, e o modelo será avaliado em um conjunto de dados real. Por fim, um método de limiarização será aplicado aos dados de aceleração para identificar trechos da via onde as irregularidades estimadas excedem os limites de segurança estabelecidos.

    \textbf{Palavras-chave}: monitoramento de ferrovias; irregularidades da via; deep learning; machine learning; simulação de multicorpos; dados de aceleração; correção da velocidade; efeito da massa; thresholding.
 \end{otherlanguage*}
\end{resumo}
% -------------------------------------------------------------
% -------------------------------------------------------------
% resumo em INGLÊS (OBRIGATÓRIO)
\begin{resumo}[Abstract]
 \begin{otherlanguage*}{english}
    This thesis proposes a deep learning approach for the reconstruction of track irregularities from acceleration data. The study first investigates the influence of the velocity and wagon mass on the acceleration measurements with the goal of normalizing the data to different operational conditions. A machine learning model will be then trained using simulated data to reconstruct track irregularities and the model will be evaluated on a real dataset. Finally, a thresholding method will be applied to the acceleration data to identify sections of track where the estimated irregularities exceed the standard safety limits.

    \textbf{Keywords}: railway track monitoring; track irregularities; deep learning; machine learning; multibody simulation; acceleration data; velocity correction; mass effect; thresholding.
 \end{otherlanguage*}
\end{resumo}
% -------------------------------------------------------------