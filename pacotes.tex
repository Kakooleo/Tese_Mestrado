% Aqui você pode adicionar seus pacotes específicos para uso em seu trabalho.
% Em PACOTES PESSOAIS insira os pacotes que desejar.
% ----------------------------------------------------------
% PACOTES BÁSICOS (ESSENCIAIS AO MODELO)
\usepackage{lmodern}			% Usa a fonte Latin Modern
\usepackage[T1]{fontenc}		% Selecao de codigos de fonte.
\usepackage[utf8]{inputenc}		% Codificacao do documento (conversão automática dos acentos)
\usepackage{lastpage}			% Usado pela Ficha catalográfica
\usepackage{indentfirst}		% Indenta o primeiro parágrafo de cada seção.
\usepackage{color,xcolor}		% Controle das cores
\usepackage{graphicx}			% Inclusão de gráficos
\usepackage{microtype} 			% para melhorias de justificação
% ----------------------------------------------------------
% ----------------------------------------------------------
% PACOTES PARA GLOSSÁRIO
\usepackage[subentrycounter,seeautonumberlist,nonumberlist=true]{glossaries}
% para usar o xindy ao invés do makeindex:
%\usepackage[xindy={language=portuguese},subentrycounter,seeautonumberlist,nonumberlist=true]{glossaries}
% ----------------------------------------------------------
% ----------------------------------------------------------
% PACOTES DE CITAÇÕES (PRINCIPAIS PACOTES DO MODELO)
\usepackage[brazilian]{backref}		% Paginas com as citações na bibl
\usepackage[alf,
	    abnt-repeated-author-omit=yes,
	    abnt-etal-list=0]{abntex2cite}		% Citações padrão ABNT
\makeatletter
\@ifpackageloaded{tocbibind}{}{\let\tocbibind\relax}
\makeatother

% É possível utilizar o sistema numérico de chamada, alterando a opção 'alf' para 'num'.
% Outros estilos bibliográficos podem ser usados. Se este for o caso, comente o pacote acima
% e utilize, por exemplo, o comando abaixo
% \bibliographystyle{acm}
% Consulte outros estilos de bibliografia consultando o manual de estilos bibliográficos do
% BibTeX em 'http://www.bibtex.org/'
% ----------------------------------------------------------
% ----------------------------------------------------------
% PACOTES ADICIONAIS (usados apenas no âmbito do Modelo Canônico do abnteX2)
\usepackage{lipsum}			% para geração de dummy text
% ----------------------------------------------------------
% ----------------------------------------------------------
% PACOTES PESSOAIS (USADOS PELO AUTOR -- acrescente aqui seus pacotes)
% ----------------------------------------------------------
\usepackage[portuguese,onelanguage]{algorithm2e}	% para inserir algoritmos (longend,vlined)
\usepackage{float}
% \usepackage{amsbsy}			% para símbolos matemáticos em negrito
% \usepackage{amscd}			% para diagramas
% \usepackage{amsfonts}			% fontes AMS
% \usepackage{amsmath}			% facilidades matemáticas
% \usepackage{amssymb}			% para os símbolos mais antigos
% \usepackage{amstext}			% para fragmentos tipo texto em modo matemático
\usepackage{amsthm}			% para teoremas
\usepackage{hyperref}			% Amplo suporte para hipertexto em LaTeX
\usepackage{cleveref}			% Referência cruzada inteligente
\usepackage{dsfont}			% para o estilo de conjuntos de números $\mathds{R}$
% \usepackage{ifthen}			% comandos de condição em LaTeX
\usepackage{listings}           	% para inserir códigos de outras linguagens de programação
% \usepackage{lscape}             	% para imprimir alguma página no formato paisagem
\usepackage{mathabx}			% conjunto de simbolos matemáticos
% \usepackage{mathrsfs}			% suporte para fontes RSFS
% \usepackage{pdfpages}           	% para inserir páginas PDF no texto

% \usepackage{verbatim}
% ----------------------------------------------------------
\usepackage{lastpage}
\usepackage{graphics}
\usepackage{float}
\usepackage{graphicx}
\usepackage{lscape}
\usepackage{longtable}
\usepackage{rotating}
\usepackage{graphics} 
\usepackage{ifthen}
\usepackage{keyval}
\usepackage{trig}
\usepackage{tabu}
\usepackage{varwidth}
% Indenta o primeiro parágrafo de cada seção.
\usepackage{indentfirst}
\usepackage{lscape}
\usepackage{array}
% Controle das cores.
\usepackage[usenames,dvipsnames]{xcolor}

% Inclusão de gráficos.
\usepackage{graphicx}

% Inclusão de páginas em PDF diretamente no documento (para uso nos apêndices).
\usepackage{pdfpages}

% Para melhorias de justificação.
\usepackage{microtype}

% Para referencias inteligentes
\usepackage[brazilian]{cleveref}

\usepackage[linesnumbered, [linesnumbered,ruled, vline,commentsnumbered]{algorithm2e}
% Citações padrão ABNT.
\usepackage[brazilian,hyperpageref]{backref}
\usepackage[alf]{abntex2cite}	
%\usepackage[num]{abntex2cite}	
\renewcommand{\backrefpagesname}{Citado na(s) página(s):~}		% Usado sem a opção hyperpageref de backref.
\renewcommand{\backref}{}										% Texto padrão antes do número das páginas.
\renewcommand*{\backrefalt}[4]{									% Define os textos da citação.
	\ifcase #1
		Nenhuma citação no texto.
	\or
		Citado na página #2.
	\else
		Citado #1 vezes nas páginas #2.
	\fi}

% \rm is deprecated and should not be used in a LaTeX2e document
% http://tex.stackexchange.com/questions/151897/always-textrm-never-rm-a-counterexample
\renewcommand{\rm}{\textrm}

% Pacotes não incluídos no template abntex2. 
% Podem ser comentados caso não queira utilizá-los.

% Inclusão de símbolos não padrão.
\usepackage{amssymb}
\usepackage{eurosym}

% Para utilizar \eqref para referenciar equações.
\usepackage{amsmath}

% Permite mostrar figuras muito largas em modo paisagem com \begin{sidewaysfigure} ao invés de \begin{figure}.
\usepackage{rotating}

% Permite customizar listas enumeradas/com marcadores.
\usepackage{enumitem}
\usepackage{enumerate}
% Permite inserir hiperlinks com \url{}.
\usepackage{bigfoot}
\usepackage{hyperref}

% Permite usar o comando \hl{} para evidenciar texto com fundo amarelo. Útil para chamar atenção a itens a fazer.
\usepackage{soulutf8}

% Colorinlistoftodos package: to insert colored comments so authors can collaborate on the content.
%\usepackage[colorinlistoftodos, textwidth=20mm, textsize=footnotesize]{todonotes}
%\newcommand{\aluno}[1]{\todo[author=\textbf{Aluno},color=green!30,caption={},inline]{#1}}
%\newcommand{\professor}[1]{\todo[author=\textbf{Professor},color=red!30,caption={},inline]{#1}}



% Permite inserir espaço em branco condicional (incluído no texto final só se necessário) em macros.
\usepackage{xspace}

% Permite incluir listagens de código com o comando \lstinputlisting{}.
\usepackage{listings}
\usepackage{caption}
\usepackage{subcaption}
\DeclareCaptionFont{white}{\color{white}}
\DeclareCaptionFormat{listing}{\colorbox{gray}{\parbox{\textwidth}{#1#2#3}}}
\captionsetup[lstlisting]{format=listing,labelfont=white,textfont=white}
\renewcommand{\lstlistingname}{Listagem}
\definecolor{mygray}{rgb}{0.5,0.5,0.5}
\lstset{
	basicstyle=\scriptsize,
	breaklines=true,
	numbers=left,
	numbersep=5pt,
	numberstyle=\tiny\color{mygray}, 
	rulecolor=\color{black},
	showstringspaces=false,
	tabsize=2,
    inputencoding=utf8,
    extendedchars=true,
    literate=%
    {é}{{\'{e}}}1
    {è}{{\`{e}}}1
    {ê}{{\^{e}}}1
    {ë}{{\¨{e}}}1
    {É}{{\'{E}}}1
    {Ê}{{\^{E}}}1
    {û}{{\^{u}}}1
    {ù}{{\`{u}}}1
    {â}{{\^{a}}}1
    {à}{{\`{a}}}1
    {á}{{\'{a}}}1
    {ã}{{\~{a}}}1
    {Á}{{\'{A}}}1
    {Â}{{\^{A}}}1
    {Ã}{{\~{A}}}1
    {ç}{{\c{c}}}1
    {Ç}{{\c{C}}}1
    {õ}{{\~{o}}}1
    {ó}{{\'{o}}}1
    {ô}{{\^{o}}}1
    {Õ}{{\~{O}}}1
    {Ó}{{\'{O}}}1
    {Ô}{{\^{O}}}1
    {î}{{\^{i}}}1
    {Î}{{\^{I}}}1
    {í}{{\'{i}}}1
    {Í}{{\~{Í}}}1
}


\newcommand{\latex}{\LaTeX\xspace}
\newcommand{\istar}{\textit{i}$^\star$\xspace}
\newcommand{\java}{Java\texttrademark\xspace}

\DeclareMathOperator*{\argmin}{arg\,min}

\usepackage{nomencl}

\usepackage{siunitx}
%\usepackage[table,xcdraw]{xcolor}


\usepackage{comment}

\usepackage{paralist}

\usepackage{caption}
\usepackage{subcaption}


\usepackage{media9}
\usepackage{pdfpages}

\usepackage[ruled,vlined,linesnumbered]{algorithm2e}


\usepackage{algpseudocode}


\usepackage{booktabs}
\usepackage{multirow}
\usepackage{ragged2e}
% \usepackage{breqn}
