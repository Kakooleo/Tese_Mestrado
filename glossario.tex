% ----------------------------------------------------------
% COMPILA O GLOSSÁRIO
\makeglossaries
% ----------------------------------------------------------
% ----------------------------------------------------------
% ENTRADAS DO GLOSSÁRIO
\newglossaryentry{pai}{
    name={pai},
    plural={pai},
    description={este é uma entrada pai, que possui outras
    subentradas.}
}
\newglossaryentry{componente}{
    name={componente},
    plural={componentes},
    parent=pai,
    description={descriação da entrada componente.}
}
\newglossaryentry{filho}{
    name={filho},
    plural={filhos},
    parent=pai,
    description={isto é uma entrada filha da entrada de nome
	\gls{pai}. Trata-se de uma entrada irmã da entrada
	\gls{componente}.
    }
}
\newglossaryentry{equilibrio}{
    name={equilíbrio da configuração},
    see=[veja também]{componente},
    description={consistência entre os \glspl{componente}}
}
\newglossaryentry{latex}{
    name={LaTeX},
    description={ferramenta de computador para autoria de
    documentos criada por D. E. Knuth}
}
\newglossaryentry{abntex2}{
    name={abnTeX2},
    see=latex,
    description={suíte para LaTeX que atende os requisitos das
    normas da ABNT para elaboração de documentos técnicos e
    científicos brasileiros}
}
% ----------------------------------------------------------
% ----------------------------------------------------------
% EXEMPLO DE CONFIGURAÇÃO DO GLOSSÁRIO
\renewcommand*{\glsseeformat}[3][\seename]{\textit{#1}  
 \glsseelist{#2}}
% ----------------------------------------------------------
% ----------------------------------------------------------
% Define nome e preâmbulo do glossário
\renewcommand{\glossaryname}{Glossário}
% \renewcommand{\glossarypreamble}{Esta é a descrição do glossário. Experimente
% visualizar outros estilos de glossários, como o \texttt{altlisthypergroup},
% por exemplo.\\
% \\}
% ----------------------------------------------------------
% ----------------------------------------------------------
% Traduções para o ambiente glossaries
\providetranslation{Glossary}{Glossário}
\providetranslation{Acronyms}{Siglas}
\providetranslation{Notation (glossaries)}{Notação}
\providetranslation{Description (glossaries)}{Descrição}
\providetranslation{Symbol (glossaries)}{Símbolo}
\providetranslation{Page List (glossaries)}{Lista de Páginas}
\providetranslation{Symbols (glossaries)}{Símbolos}
\providetranslation{Numbers (glossaries)}{Números} 
% ----------------------------------------------------------
% ----------------------------------------------------------
% Estilo de glossário
\setglossarystyle{index}
% \setglossarystyle{altlisthypergroup}
% \setglossarystyle{tree}
% ----------------------------------------------------------